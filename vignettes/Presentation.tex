\documentclass{beamer}\usepackage[]{graphicx}\usepackage[]{color}
%% maxwidth is the original width if it is less than linewidth
%% otherwise use linewidth (to make sure the graphics do not exceed the margin)
\makeatletter
\def\maxwidth{ %
  \ifdim\Gin@nat@width>\linewidth
    \linewidth
  \else
    \Gin@nat@width
  \fi
}
\makeatother

\definecolor{fgcolor}{rgb}{0.345, 0.345, 0.345}
\newcommand{\hlnum}[1]{\textcolor[rgb]{0.686,0.059,0.569}{#1}}%
\newcommand{\hlstr}[1]{\textcolor[rgb]{0.192,0.494,0.8}{#1}}%
\newcommand{\hlcom}[1]{\textcolor[rgb]{0.678,0.584,0.686}{\textit{#1}}}%
\newcommand{\hlopt}[1]{\textcolor[rgb]{0,0,0}{#1}}%
\newcommand{\hlstd}[1]{\textcolor[rgb]{0.345,0.345,0.345}{#1}}%
\newcommand{\hlkwa}[1]{\textcolor[rgb]{0.161,0.373,0.58}{\textbf{#1}}}%
\newcommand{\hlkwb}[1]{\textcolor[rgb]{0.69,0.353,0.396}{#1}}%
\newcommand{\hlkwc}[1]{\textcolor[rgb]{0.333,0.667,0.333}{#1}}%
\newcommand{\hlkwd}[1]{\textcolor[rgb]{0.737,0.353,0.396}{\textbf{#1}}}%

\usepackage{framed}
\makeatletter
\newenvironment{kframe}{%
 \def\at@end@of@kframe{}%
 \ifinner\ifhmode%
  \def\at@end@of@kframe{\end{minipage}}%
  \begin{minipage}{\columnwidth}%
 \fi\fi%
 \def\FrameCommand##1{\hskip\@totalleftmargin \hskip-\fboxsep
 \colorbox{shadecolor}{##1}\hskip-\fboxsep
     % There is no \\@totalrightmargin, so:
     \hskip-\linewidth \hskip-\@totalleftmargin \hskip\columnwidth}%
 \MakeFramed {\advance\hsize-\width
   \@totalleftmargin\z@ \linewidth\hsize
   \@setminipage}}%
 {\par\unskip\endMakeFramed%
 \at@end@of@kframe}
\makeatother

\definecolor{shadecolor}{rgb}{.97, .97, .97}
\definecolor{messagecolor}{rgb}{0, 0, 0}
\definecolor{warningcolor}{rgb}{1, 0, 1}
\definecolor{errorcolor}{rgb}{1, 0, 0}
\newenvironment{knitrout}{}{} % an empty environment to be redefined in TeX

\usepackage{alltt} % class of presentation document
\usepackage{Sweave} % sweave package
\usepackage{beamerthemebars} % creates the pretty bars and links
\IfFileExists{upquote.sty}{\usepackage{upquote}}{}

\begin{document}

\title{Joint Bayesian estimatation of diets from Stable Isotopes and Fatty Acid compositions}
\author{Phil}

\frame{\titlepage} % create title page

\section{Estimating diet proportions}


\begin{frame}

You are what you eat. Almost. Take Stable Isotope $i$ of a predator preying on $j=1...J$ prey species:

\[
  SI_{i}=\sum_{j} p_j SI_{i,j}
\]

This would be true if stable isotopes for all preys were incorporated 1:1, but some isotopes are preferentially assimilated. Furthermore, different diet items have different amounts of each element (i.e., N in protein). This leads to:

\[
  SI_{i}=\sum_{j} p_j (SI_{i,j} + \delta_{i,j} + \beta_{i,j} SI_{i,j})
\]

\end{frame}

\begin{frame}

This is essentially a regression model, and we can estiamte parameters $\beta_{i,j}$ and $\delta_{i,j}$ from controlled experiments in which $p_j$ is known...which lets us estiamte $p_j$ in a diet study as the only unknown.\\

\quad

Unfortunatelly, Fatty Acids are more complicated, as they are measured as percentages, or more precisely, a composition. These have funny properties: the most obvious one is that they have a sum constraint: 

\[
  \sum_f FA_f = 1
\]


\end{frame}

\begin{frame}

In the theory of 'you are what you eat', this means that:

\[
  FA_{f}=C(\sum_{j} p_j FA_{f,j})
\]

where C is a closure operation enforcing the sum to one constraint.\\

\end{frame}

\begin{frame}

Again, it is not all that simple since different prey species do not have the same fat content $(\phi)$ to start with, such that eating a kg of one is not the same as eating a kg of another. \\

\quad

Also, FAs are differentally assimilated, and some are metabolised to some extent, resulting in only a relative proportion $\Delta_{f,j}$, where again $\sum_f \Delta_{f,j} = 1$. This leads to 

\[
  FA_{f}=C(\sum_{j} p_j [FA_{f,j} \Delta_{f,j} \phi_{j}])
\]

\end{frame}

% this section will show the intialization code

\begin{frame}

Again, $\Delta_{f,j}$ can be estiamted from controlled experiments (it's a bit more tricky here because of the compositional nature...), and $\phi_{j}$ can be emasured along with fatty acid profiles, or can be found in literature.\\
\quad

The session today follows two tutorials:
\begin{itemize}
\item a simulated example (in the vignettes folder) showing how Bayesian methods can be used to estiamte $p$
\item an analysis of experimental data from a 2006 squid diet paper (Stowasser et al.), which first estiamtes the deltas for both markers from single diet treatments and then estiamtes diet proportions for animals in a mixed food treatment. 

\end{itemize}

Both will make up the demonstration of the method in the final paper...

\end{frame}

\end{document}
