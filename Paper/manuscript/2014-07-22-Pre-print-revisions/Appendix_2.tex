\documentclass[12pt]{article} 
%\geometry{verbose,letterpaper,tmargin=2.54cm,bmargin=2.54cm,lmargin=2.54cm,rmargin=2.54cm} 
\usepackage{setspace}%
\usepackage{lineno}%
\usepackage{authblk}%
\usepackage[parfill]{parskip}
\usepackage{hyperref}
\usepackage{graphicx}
\usepackage{float}
\usepackage{rotating}
\usepackage{mathtools}%
\usepackage[backend=biber,style= ele,natbib=true]{biblatex}
\addbibresource{Stable_Isotopes_and_Fatty_acids.bib} 

\makeatletter%
\def\@maketitle{%
  \vskip 2em%
  \begin{center}%
%  \let \footnote \thanks
    {\Large\bfseries \@title \par}%
    \vskip 1.5em%
    {\normalsize
      \lineskip .5em%
      \begin{tabular}[t]{c}%
        \@author
      \end{tabular}\par}%
    \vskip 1em%
    {\normalsize \@date}%
  \end{center}%
  \par
  \vskip 1.5em}
\makeatother

\begin{document}

\renewcommand\footnotemark{}

\author{Philipp Neubauer*}\thanks{*Corresponding author electronic address: neubauer.phil@gmail.com}%
\affil{Dragonfly Science,\\PO Box 27535, Wellington 6141, New Zealand}%

\author{Olaf P. Jensen}%
\affil{Institute of Marine and Coastal Sciences\\Rutgers University, New Brunswick, NJ 08901, USA}%


\title{Selecting fatty acids for diet analysis: an ordination approach}
\maketitle

Given the data transformations applied in our model, choosing a large
number of fatty acids (FAs) becomes computationally
prohibitive. This is in part due to the fully Bayesian implementation,
and en empirical Bayes method could provide more flexibility in the
future. Although model complexity scales with the number of FAs,
individual predators and prey species, transformation of FAs disproportionately affect computation time. It is thus inevitable to
choose an appropriate subset of FAs for diet analysis. When FA
profiles are obtained (typically from gas chromatography), the
practitioner faces the choice of which FAs out of the potentially
large number of measured FAs to retain for the analysis of diet
proportions. Typically, only some fatty acids will be informative about diets by separating sources in multivariate space. Adding FAs beyond these informative FAs does not improve estimates of diet proportions, but may instead increase co-linearity among prey samples..

While most studies quantify and list the most abundant FAs, these may
not be the most informative to discriminate among potential prey
species. Choosing FAs with experimentally validated conversion
coefficients is another important consideration. Eliminating FAs with
conversion coefficients that are unknown and suspected to be far from
1 is an important first step since their inclusion can introduce
significant uncertainty and error in point estimates. Once this
preliminary sorting is complete, we propose to select variables based
on their contribution to axes in a constrained ordination. We
specifically use Constrained Analysis of Principal Coordinates
\citep{anderson_canonical_2003} since it can deal with any distance
metric, and use compositional distance as a distance metric for
ordination \cite{aitchison_logratio_2000}. For each FA $f$, we sum over
the product of the FA contribution to the ordination axes and the
axes’ respective eigenvalues: $A_f = \sum_{a=1}^{n-1} \lambda_a
c_{f,a}$, where $a$ indexes individual ordination axes, and $c_{f,a}$ is the
contribution of FA p$f$ to axis $a$. Each $A_f$ then contributes a
proportion $p$ to $A=\sum_f A_f$, and we can sort $A_f$ and choose a
number of variables that contribute to a cumulative proportion $P$ of the
cumulative separation $A$.


\printbibliography

\end{document}